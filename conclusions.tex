\chapter{Conclusions}

We introduced SDCEL, a scalable approach to compute the overlay operation among two layers that represent polygons from a planar subdivision of a surface. 
Both input layers use the DCEL edge-list data structure to store their polygons. 
We first consider inputs with regular polygons. Existing sequential DCEL overlay implementations fail for large datasets. We first presented two partition strategies that guarantee that each partition collects the required data from each layer to work independently. 
We also proposed several optimizations to improve performance. Our experimental evaluation using real datasets shows that SDCEL has very good scale-up and speed-up performance and can compute the overlay over very large layers (up to 37M edges) in a few seconds. 
We also considered inputs with polygons that may contain scattered line segments also known as dangle and cut edges. For this case we presented a scalable polygonization method. 

We also presented a novel, scalable approach for discovering moving flock patterns in large trajectory databases. By leveraging distributed frameworks, the proposed method overcomes the limitations of sequential algorithms that struggle with large-scale spatio-temporal datasets. Through partitioning and replication, as well as improvements in pruning and temporal joins, this approach efficiently handles dense data, offering significant performance improvements over traditional methods. The evaluation results demonstrate the scalability and effectiveness of the approach, making it a valuable contribution for analyzing complex movement patterns.

% Include geoinformatica paper
% Add in related work of Jensen ICPE and others...
