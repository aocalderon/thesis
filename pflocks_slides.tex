\documentclass{beamer}
\usepackage{graphicx}
\usepackage{amsmath}

\title{Scalable Processing of Moving Flock Patterns}
\author{Andres Oswaldo Calderon Romero}
\institute{University of California, Riverside}
\date{November 2024}

\begin{document}

\frame{\titlepage}

\section{Introduction}
\begin{frame}{Introduction}
    \begin{itemize}
        \item Advances in spatio-temporal data collection have increased the need to process large datasets.
        \item Flock patterns represent groups moving closely together over time.
        \item Applications include transportation, ecology, and urban planning.
    \end{itemize}
\end{frame}

\section{Flock Pattern Detection}
\begin{frame}{Flock Pattern Detection}
    \begin{itemize}
        \item A flock pattern is a group of entities within a given radius for a specified period.
        \item Detection is computationally intensive, especially with large datasets.
        \item Current methods (e.g., BFE) lack scalability.
    \end{itemize}
\end{frame}

\section{Sequential Approaches}
\begin{frame}{Sequential Approaches}
    \begin{itemize}
        \item \textbf{BFE Algorithm:} Basic Flock Evaluation identifies maximal disks, or regions where entities move closely.
        \item \textbf{PSI Algorithm:} Optimizes BFE by reducing the search space through a half-square approach.
        \item PSI generally outperforms BFE on large datasets.
    \end{itemize}
\end{frame}

\section{Challenges in Sequential Approaches}
\begin{frame}{Challenges in Sequential Approaches}
    \begin{itemize}
        \item Sequential methods suffer from high computational demands and scalability issues.
        \item Spatial and temporal data distribution creates performance bottlenecks.
    \end{itemize}
\end{frame}

\section{Scalable Solutions}
\begin{frame}{Scalable Solutions}
    \begin{itemize}
        \item \textbf{Partitioning Strategy:} Quadtree-based partitioning for spatially distributed processing.
        \item \textbf{Replication and Safe Zones:} Ensures correct flock detection across partitions.
        \item \textbf{Temporal Joins:} Strategies to efficiently handle flocks moving across partitions.
    \end{itemize}
\end{frame}

\section{Distributed Temporal Join Strategies}
\begin{frame}{Distributed Temporal Join Strategies}
    \begin{itemize}
        \item \textbf{Master, By-Level, LCA, Cube-based} methods evaluated for handling partitioned flocks.
        \item Cube-based approach performs best for large datasets, leveraging higher parallelism.
    \end{itemize}
\end{frame}

\section{Experimental Evaluation}
\begin{frame}{Experimental Evaluation}
    \begin{itemize}
        \item Tested on synthetic datasets to evaluate partitioning and temporal join efficiency.
        \item \textbf{Results:} PSI outperformed BFE, and Cube-based approach demonstrated best scalability.
    \end{itemize}
\end{frame}

\section{Optimizations}
\begin{frame}{Optimizations}
    \begin{itemize}
        \item Optimal partition size and configuration improve performance.
        \item Local, intermediate, and global reduce phases enhance processing efficiency.
    \end{itemize}
\end{frame}

\section{Conclusions}
\begin{frame}{Conclusions}
    \begin{itemize}
        \item The scalable approach to flock pattern detection overcomes limitations of traditional methods.
        \item Distributed processing using partitioning and replication significantly improves efficiency.
        \item Results highlight effective strategies for handling large spatio-temporal datasets.
    \end{itemize}
\end{frame}

\end{document}
